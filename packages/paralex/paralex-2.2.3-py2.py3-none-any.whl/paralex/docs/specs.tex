
This is a human-readable rendition of a JSON file defining a
frictionless package. It was generated automatically.

\begin{itemize}
\item
  \texttt{name}: paralex
\item
  \texttt{licenses}:

  \begin{itemize}
  \tightlist
  \item
    \href{https://opensource.org/licenses/GPL-3.0}{GPL-3.0: GNU General
    Public License 3.0}
  \end{itemize}
\item
  \texttt{keywords}: lexicon, inflection, linguistics, morphology,
  paradigms
\item
  \texttt{profile} data-package
\item
  \texttt{contributors}

  \begin{itemize}
  \tightlist
  \item
    {[}1{]}

    \begin{itemize}
    \tightlist
    \item
      \texttt{title} Sacha Beniamine
    \item
      \texttt{role} maintainer
    \end{itemize}
  \item
    {[}2{]}

    \begin{itemize}
    \tightlist
    \item
      \texttt{title} Cormac Anderson
    \item
      \texttt{role} contributor
    \end{itemize}
  \item
    {[}3{]}

    \begin{itemize}
    \tightlist
    \item
      \texttt{title} Jules Bouton
    \item
      \texttt{role} contributor
    \end{itemize}
  \item
    {[}4{]}

    \begin{itemize}
    \tightlist
    \item
      \texttt{title} Mae Carroll
    \item
      \texttt{role} contributor
    \end{itemize}
  \item
    {[}5{]}

    \begin{itemize}
    \tightlist
    \item
      \texttt{title} Borja Herce
    \item
      \texttt{role} contributor
    \end{itemize}
  \item
    {[}6{]}

    \begin{itemize}
    \tightlist
    \item
      \texttt{title} Matías Guzmán Naranjo
    \item
      \texttt{role} contributor
    \end{itemize}
  \item
    {[}7{]}

    \begin{itemize}
    \tightlist
    \item
      \texttt{title} Matteo Pellegrini
    \item
      \texttt{role} contributor
    \end{itemize}
  \item
    {[}8{]}

    \begin{itemize}
    \tightlist
    \item
      \texttt{title} Erich Round
    \item
      \texttt{role} contributor
    \end{itemize}
  \item
    {[}9{]}

    \begin{itemize}
    \tightlist
    \item
      \texttt{title} Helen Sims-Williams
    \item
      \texttt{role} contributor
    \end{itemize}
  \end{itemize}
\item
  \texttt{version} 2.0.8
\item
  \texttt{languages\_iso639} {[}`'{]}
\item
  \texttt{related\_identifiers}

  \begin{itemize}
  \tightlist
  \item
    {[}1{]}

    \begin{itemize}
    \tightlist
    \item
      \texttt{identifier}
    \item
      \texttt{relation} \textless Any of isCitedBy, cites,
      isSupplementTo, isSupplementedBy, isContinuedBy, continues,
      isDescribedBy, describes, hasMetadata, isMetadataFor,
      isNewVersionOf, isPreviousVersionOf, isPartOf, hasPart,
      isReferencedBy, references, isDocumentedBy, documents,
      isCompiledBy, compiles, isVariantFormOf, isOriginalFormof,
      isIdenticalTo, isAlternateIdentifier, isReviewedBy, reviews,
      isDerivedFrom, isSourceOf, requires, isRequiredBy, isObsoletedBy,
      obsoletes\textgreater{}
    \end{itemize}
  \end{itemize}
\end{itemize}

This package describes the following files:

\hypertarget{forms}{%
\subsection{\texorpdfstring{\textbf{forms}}{forms}}\label{forms}}

Inflected forms - This file is located in \texttt{forms.csv}.

\begin{itemize}
\item
  The identifier column (or \texttt{primaryKey}) is
  \texttt{{[}\textquotesingle{}form\_id\textquotesingle{}{]}}
\item
  \textbf{Formal relations (foreignKeys)} with other tables:

  \begin{itemize}
  \tightlist
  \item
    Each value in column
    \texttt{{[}\textquotesingle{}cell\textquotesingle{}{]}} of forms
    must refer to
    \texttt{{[}\textquotesingle{}cell\_id\textquotesingle{}{]}} in table
    \texttt{cells}
  \item
    Each value in column
    \texttt{{[}\textquotesingle{}lexeme\textquotesingle{}{]}} of forms
    must refer to
    \texttt{{[}\textquotesingle{}lexeme\_id\textquotesingle{}{]}} in
    table \texttt{lexemes}
  \end{itemize}
\end{itemize}

\textbf{Columns} defined by \texttt{forms-schema}:

\begin{itemize}
\item
  \textbf{\texttt{form\_id}} (\texttt{string}): Form table row
  identifiers. These identifiers are specific to form, lexeme, cell
  triples.

  \begin{itemize}
  \tightlist
  \item
    constraints: a \texttt{form\_id} is obligatory; it must be unique.
  \end{itemize}
\item
  \textbf{\texttt{lexeme}} (\texttt{string}): Reference to a lexeme
  identifier. Lexeme identifiers must be unique to paradigms.

  \begin{itemize}
  \item
    constraints: a \texttt{lexeme} is obligatory.
  \item
    \texttt{rdfProperty}:
    \url{https://www.paralex-standard.org/paralex_ontology.xml\#lexeme}
  \end{itemize}
\item
  \textbf{\texttt{cell}} (\texttt{string}): Reference to a cell
  identifier. The set of feature values as would appear in a gloss,
  separated by dots, eg. prs.ind.1sg or f.pl

  \begin{itemize}
  \item
    constraints: a \texttt{cell} is obligatory.
  \item
    \texttt{rdfProperty}:
    \url{https://www.paralex-standard.org/paralex_ontology.xml\#cell}
  \end{itemize}
\item
  \textbf{\texttt{phon\_form}} (\texttt{string}): Inflected form
  (phonemic or phonetic). The form, given in phonemic or phonetic
  notation, with sounds separated by spaces

  \begin{itemize}
  \tightlist
  \item
    \texttt{rdfProperty}:
    \url{https://www.paralex-standard.org/paralex_ontology.xml\#phon_form}
  \item
    \texttt{missingValues}: \texttt{\#DEF\#}
  \end{itemize}
\item
  \textbf{\texttt{orth\_form}} (\texttt{string}): Inflected form
  (orthographic). The form, given orthographically

  \begin{itemize}
  \tightlist
  \item
    \texttt{rdfProperty}:
    \url{https://www.paralex-standard.org/paralex_ontology.xml\#orth_form}
  \item
    \texttt{missingValues}: \texttt{\#DEF\#}
  \end{itemize}
\item
  \textbf{\texttt{analysed\_phon\_form}} (\texttt{string}): Inflected
  form with analysis, such as segmentation markers (phonemic or
  phonetic). The form, given in phonemic or phonetic notation, with
  sounds separated by spaces, and analysis markers.

  \begin{itemize}
  \tightlist
  \item
    \texttt{rdfProperty}:
    \url{https://www.paralex-standard.org/paralex_ontology.xml\#analysed_phon_form}
  \item
    \texttt{missingValues}: \texttt{\#DEF\#}
  \end{itemize}
\item
  \textbf{\texttt{analysed\_orth\_form}} (\texttt{string}): Inflected
  form with analysis, such as segmentation markers (orthographic). The
  form, given orthographically, with markers for analysis.

  \begin{itemize}
  \tightlist
  \item
    \texttt{rdfProperty}:
    \url{https://www.paralex-standard.org/paralex_ontology.xml\#analysed_orth_form}
  \item
    \texttt{missingValues}: \texttt{\#DEF\#}
  \end{itemize}
\item
  \textbf{\texttt{frequency}} (\texttt{number}): Frequency. Frequency
  for this row.

  \begin{itemize}
  \tightlist
  \item
    \texttt{rdfProperty}:
    \url{https://www.paralex-standard.org/paralex_ontology.xml\#frequency}
  \end{itemize}
\item
  \textbf{\texttt{analysis\_tag}} (\texttt{string}): Tags for marking
  separate analyses. Identifies sets of forms which are related by the
  same analysis. Eg: forms from two distinct sources, or a more phonetic
  and a more phonological transcription.

  \begin{itemize}
  \tightlist
  \item
    \texttt{rdfProperty}:
    \url{https://www.paralex-standard.org/paralex_ontology.xml\#analysis_tag}
  \end{itemize}
\item
  \textbf{\texttt{defectiveness\_tag}} (\texttt{string}): Tags for
  defectiveness status. Identifies sets of defective forms (eg. pluralia
  tantum).

  \begin{itemize}
  \tightlist
  \item
    \texttt{rdfProperty}:
    \url{https://www.paralex-standard.org/paralex_ontology.xml\#defectiveness_tag}
  \end{itemize}
\item
  \textbf{\texttt{epistemic\_tag}} (\texttt{string}): Tags for epistemic
  status. Identifies sets of forms with the same epistemic status.

  \begin{itemize}
  \tightlist
  \item
    \texttt{rdfProperty}:
    \url{https://www.paralex-standard.org/paralex_ontology.xml\#epistemic_tag}
  \end{itemize}
\item
  \textbf{\texttt{variants\_tag}} (\texttt{string}): Tags for form
  variants. Identifies sets of forms used by specific groups of
  speakers. Eg. dialectal variants.

  \begin{itemize}
  \tightlist
  \item
    \texttt{rdfProperty}:
    \url{https://www.paralex-standard.org/paralex_ontology.xml\#variants_tag}
  \end{itemize}
\item
  \textbf{\texttt{overabundance\_tag}} (\texttt{string}): Tags for
  overabundant forms. Identifies sets of overabundant forms. For
  example, overabundant forms across lexemes might belong to a series of
  regular and irregular forms, or a series of short and long forms, etc.

  \begin{itemize}
  \tightlist
  \item
    \texttt{rdfProperty}:
    \url{https://www.paralex-standard.org/paralex_ontology.xml\#overabundance_tag}
  \end{itemize}
\item
  \textbf{\texttt{comment}} (\texttt{string}): Comment. Human-readable
  comment.

  \begin{itemize}
  \tightlist
  \item
    \texttt{rdfProperty}:
    \url{http://www.w3.org/2000/01/rdf-schema\#comment}
  \end{itemize}
\item
  \textbf{\texttt{source}} (\texttt{string}): Source. Reference to a
  specific source (bibtex key). If used, the dataset should comprise a
  .bib file where the keys are referenced.

  \begin{itemize}
  \tightlist
  \item
    \texttt{rdfProperty}:
    \url{https://www.paralex-standard.org/paralex_ontology.xml\#source}
  \end{itemize}
\item
  \textbf{\texttt{POS}} (\texttt{string}): Part of Speech. The relevant
  part of speech for this item. This must refer to a PartOfSpeech entity
  from the lexinfo (https://lexinfo.net/) ontology.

  \begin{itemize}
  \item
    constraints: a \texttt{POS} must be one of the values:
    \texttt{verb}, \texttt{numeral}, \texttt{conjunction},
    \texttt{noun}, \texttt{adposition}, \texttt{determiner},
    \texttt{article}, \texttt{adverb}, \texttt{pronoun},
    \texttt{fusedPreposition}, \texttt{adjective}, \texttt{symbol},
    \texttt{particle}, \texttt{conditionalParticle},
    \texttt{demonstrativePronoun}, \texttt{interjection},
    \texttt{semiColon}, \texttt{diminutiveNoun},
    \texttt{possessivePronoun}, \texttt{prepositionalAdverb},
    \texttt{compoundPreposition}, \texttt{interrogativeRelativePronoun},
    \texttt{possessiveParticle}, \texttt{plainVerb}, \texttt{letter},
    \texttt{interrogativeDeterminer}, \texttt{relativePronoun},
    \texttt{postposition}, \texttt{fusedPronounAuxiliary},
    \texttt{interrogativeOrdinalNumeral},
    \texttt{indefiniteOrdinalNumeral}, \texttt{strongPersonalPronoun},
    \texttt{possessiveRelativePronoun}, \texttt{ordinalAdjective},
    \texttt{collectivePronoun}, \texttt{commonNoun},
    \texttt{infinitiveParticle}, \texttt{comparativeParticle},
    \texttt{partitiveArticle}, \texttt{invertedComma},
    \texttt{lightVerb}, \texttt{emphaticPronoun},
    \texttt{distinctiveParticle}, \texttt{genericNumeral},
    \texttt{possessiveAdjective}, \texttt{reflexivePossessivePronoun},
    \texttt{colon}, \texttt{coordinationParticle},
    \texttt{presentParticipleAdjective},
    \texttt{fusedPrepositionPronoun}, \texttt{cardinalNumeral},
    \texttt{indefiniteDeterminer}, \texttt{numeralFraction},
    \texttt{questionMark}, \texttt{generalAdverb},
    \texttt{superlativeParticle}, \texttt{point},
    \texttt{indefiniteMultiplicativeNumeral}, \texttt{comma},
    \texttt{closeParenthesis}, \texttt{futureParticle},
    \texttt{personalPronoun}, \texttt{reflexivePersonalPronoun},
    \texttt{adverbialPronoun}, \texttt{reciprocalPronoun},
    \texttt{openParenthesis}, \texttt{pastParticipleAdjective},
    \texttt{negativePronoun}, \texttt{relativeDeterminer},
    \texttt{existentialPronoun}, \texttt{pronominalAdverb},
    \texttt{relativeParticle}, \texttt{exclamativeDeterminer},
    \texttt{multiplicativeNumeral}, \texttt{reflexiveDeterminer},
    \texttt{modal}, \texttt{unclassifiedParticle}, \texttt{properNoun},
    \texttt{allusivePronoun}, \texttt{interrogativeCardinalNumeral},
    \texttt{bullet}, \texttt{subordinatingConjunction},
    \texttt{irreflexivePersonalPronoun}, \texttt{possessiveDeterminer},
    \texttt{negativeParticle}, \texttt{indefinitePronoun},
    \texttt{generalizationWord}, \texttt{coordinatingConjunction},
    \texttt{deficientVerb}, \texttt{adjective-i},
    \texttt{impersonalPronoun}, \texttt{indefiniteCardinalNumeral},
    \texttt{adjective-na}, \texttt{qualifierAdjective},
    \texttt{affirmativeParticle}, \texttt{mainVerb},
    \texttt{fusedPrepositionDeterminer}, \texttt{indefiniteArticle},
    \texttt{weakPersonalPronoun}, \texttt{suspensionPoints},
    \texttt{interrogativeMultiplicativeNumeral},
    \texttt{affixedPersonalPronoun}, \texttt{auxiliary},
    \texttt{circumposition}, \texttt{copula},
    \texttt{demonstrativeDeterminer}, \texttt{participleAdjective},
    \texttt{exclamativePoint}, \texttt{interrogativePronoun},
    \texttt{presentativePronoun}, \texttt{punctuation},
    \texttt{definiteArticle}, \texttt{slash},
    \texttt{exclamativePronoun}, \texttt{preposition},
    \texttt{conditionalPronoun}, \texttt{relationNoun},
    \texttt{interrogativeParticle}.
  \item
    \texttt{rdfProperty}:
    \url{https://www.paralex-standard.org/paralex_ontology.xml\#POS}
  \end{itemize}
\end{itemize}

\hypertarget{frequencies}{%
\subsection{\texorpdfstring{\textbf{frequencies}}{frequencies}}\label{frequencies}}

Frequencies - This file is located in \texttt{frequencies.csv}.

\begin{itemize}
\item
  The identifier column (or \texttt{primaryKey}) is
  \texttt{{[}\textquotesingle{}freq\_id\textquotesingle{}{]}}
\item
  \textbf{Formal relations (foreignKeys)} with other tables:

  \begin{itemize}
  \tightlist
  \item
    Each value in column
    \texttt{{[}\textquotesingle{}cell\textquotesingle{}{]}} of
    frequencies must refer to
    \texttt{{[}\textquotesingle{}cell\_id\textquotesingle{}{]}} in table
    \texttt{cells}
  \item
    Each value in column
    \texttt{{[}\textquotesingle{}form\textquotesingle{}{]}} of
    frequencies must refer to
    \texttt{{[}\textquotesingle{}form\_id\textquotesingle{}{]}} in table
    \texttt{forms}
  \item
    Each value in column
    \texttt{{[}\textquotesingle{}lexeme\textquotesingle{}{]}} of
    frequencies must refer to
    \texttt{{[}\textquotesingle{}lexeme\_id\textquotesingle{}{]}} in
    table \texttt{lexemes}
  \end{itemize}
\end{itemize}

\textbf{Columns} defined by \texttt{frequencies-schema}:

\begin{itemize}
\item
  \textbf{\texttt{freq\_id}} (\texttt{string}): Frequency record
  identifier. One frequency value, for a single data point and source

  \begin{itemize}
  \tightlist
  \item
    constraints: a \texttt{freq\_id} is obligatory; it must be unique.
  \end{itemize}
\item
  \textbf{\texttt{form}} (\texttt{string}): Reference to form table row
  identifiers. These identifiers are specific to form, lexeme, cell
  triples.

  \begin{itemize}
  \tightlist
  \item
    \texttt{rdfProperty}:
    \url{https://www.paralex-standard.org/paralex_ontology.xml\#form}
  \end{itemize}
\item
  \textbf{\texttt{lexeme}} (\texttt{string}): Reference to a lexeme
  identifier. Lexeme identifiers must be unique to paradigms.

  \begin{itemize}
  \tightlist
  \item
    \texttt{rdfProperty}:
    \url{https://www.paralex-standard.org/paralex_ontology.xml\#lexeme}
  \end{itemize}
\item
  \textbf{\texttt{cell}} (\texttt{string}): Reference to a cell
  identifier. The set of feature values as would appear in a gloss,
  separated by dots, eg. prs.ind.1sg or f.pl

  \begin{itemize}
  \tightlist
  \item
    \texttt{rdfProperty}:
    \url{https://www.paralex-standard.org/paralex_ontology.xml\#cell}
  \end{itemize}
\item
  \textbf{\texttt{frequency}} (\texttt{number}): Frequency. Frequency
  for this row.

  \begin{itemize}
  \tightlist
  \item
    \texttt{rdfProperty}:
    \url{https://www.paralex-standard.org/paralex_ontology.xml\#frequency}
  \end{itemize}
\item
  \textbf{\texttt{analysis\_tag}} (\texttt{string}): Tags for marking
  separate analyses. Identifies sets of forms which are related by the
  same analysis. Eg: forms from two distinct sources, or a more phonetic
  and a more phonological transcription.

  \begin{itemize}
  \tightlist
  \item
    \texttt{rdfProperty}:
    \url{https://www.paralex-standard.org/paralex_ontology.xml\#analysis_tag}
  \end{itemize}
\item
  \textbf{\texttt{defectiveness\_tag}} (\texttt{string}): Tags for
  defectiveness status. Identifies sets of defective forms (eg. pluralia
  tantum).

  \begin{itemize}
  \tightlist
  \item
    \texttt{rdfProperty}:
    \url{https://www.paralex-standard.org/paralex_ontology.xml\#defectiveness_tag}
  \end{itemize}
\item
  \textbf{\texttt{epistemic\_tag}} (\texttt{string}): Tags for epistemic
  status. Identifies sets of forms with the same epistemic status.

  \begin{itemize}
  \tightlist
  \item
    \texttt{rdfProperty}:
    \url{https://www.paralex-standard.org/paralex_ontology.xml\#epistemic_tag}
  \end{itemize}
\item
  \textbf{\texttt{variants\_tag}} (\texttt{string}): Tags for form
  variants. Identifies sets of forms used by specific groups of
  speakers. Eg. dialectal variants.

  \begin{itemize}
  \tightlist
  \item
    \texttt{rdfProperty}:
    \url{https://www.paralex-standard.org/paralex_ontology.xml\#variants_tag}
  \end{itemize}
\item
  \textbf{\texttt{overabundance\_tag}} (\texttt{string}): Tags for
  overabundant forms. Identifies sets of overabundant forms. For
  example, overabundant forms across lexemes might belong to a series of
  regular and irregular forms, or a series of short and long forms, etc.

  \begin{itemize}
  \tightlist
  \item
    \texttt{rdfProperty}:
    \url{https://www.paralex-standard.org/paralex_ontology.xml\#overabundance_tag}
  \end{itemize}
\item
  \textbf{\texttt{source}} (\texttt{string}): Source. Reference to a
  specific source (bibtex key). If used, the dataset should comprise a
  .bib file where the keys are referenced.

  \begin{itemize}
  \tightlist
  \item
    \texttt{rdfProperty}:
    \url{https://www.paralex-standard.org/paralex_ontology.xml\#source}
  \end{itemize}
\end{itemize}

\hypertarget{sounds}{%
\subsection{\texorpdfstring{\textbf{sounds}}{sounds}}\label{sounds}}

Sound inventory with distinctive features - This file is located in
\texttt{sounds.csv}.

\begin{itemize}
\tightlist
\item
  The identifier column (or \texttt{primaryKey}) is
  \texttt{{[}\textquotesingle{}sound\_id\textquotesingle{}{]}}
\item
  \texttt{missingValues}: ``
\end{itemize}

\textbf{Columns} defined by \texttt{sounds-schema}:

\begin{itemize}
\item
  \textbf{\texttt{sound\_id}} (\texttt{string}): sound representation.
  These identifiers are specific to sounds.

  \begin{itemize}
  \tightlist
  \item
    constraints: a \texttt{sound\_id} is obligatory; it must be unique.
  \end{itemize}
\item
  \textbf{\texttt{label}} (\texttt{string}): label for this row. A human
  readable label for the row.

  \begin{itemize}
  \tightlist
  \item
    \texttt{rdfProperty}:
    \url{http://www.w3.org/2000/01/rdf-schema\#label}
  \end{itemize}
\item
  \textbf{\texttt{comment}} (\texttt{string}): Comment. Human-readable
  comment.

  \begin{itemize}
  \tightlist
  \item
    \texttt{rdfProperty}:
    \url{http://www.w3.org/2000/01/rdf-schema\#comment}
  \end{itemize}
\item
  \textbf{\texttt{CLTS\_id}} (\texttt{string}): Identifier of this sound
  in CLTS. Reference to this sound in CLTS data.

  \begin{itemize}
  \item
    constraints: a \texttt{CLTS\_id} must be unique.
  \item
    \texttt{rdfProperty}:
    \url{https://www.paralex-standard.org/paralex_ontology.xml\#CLTS_id}
  \end{itemize}
\item
  \textbf{\texttt{PHOIBLE\_id}} (\texttt{string}): Identifier of this
  sound in PHOIBLE. Reference to this sound in PHOIBLE.

  \begin{itemize}
  \item
    constraints: a \texttt{PHOIBLE\_id} must be unique.
  \item
    \texttt{rdfProperty}:
    \url{https://www.paralex-standard.org/paralex_ontology.xml\#PHOIBLE_id}
  \end{itemize}
\end{itemize}

\hypertarget{graphemes}{%
\subsection{\texorpdfstring{\textbf{graphemes}}{graphemes}}\label{graphemes}}

Graphemes inventory - This file is located in \texttt{graphemes.csv}.

\begin{itemize}
\tightlist
\item
  The identifier column (or \texttt{primaryKey}) is
  \texttt{{[}\textquotesingle{}grapheme\_id\textquotesingle{}{]}}
\item
  \texttt{missingValues}: ``
\end{itemize}

\textbf{Columns} defined by \texttt{graphemes-schema}:

\begin{itemize}
\item
  \textbf{\texttt{grapheme\_id}} (\texttt{string}): grapheme
  representation. These identifiers are specific to graphemes.

  \begin{itemize}
  \item
    constraints: a \texttt{grapheme\_id} is obligatory; it must be
    unique.
  \item
    \texttt{rdfType}:
    \url{http://linguistics-ontology.org/gold/2010/Grapheme}
  \end{itemize}
\item
  \textbf{\texttt{comment}} (\texttt{string}): Comment. Human-readable
  comment.

  \begin{itemize}
  \tightlist
  \item
    \texttt{rdfProperty}:
    \url{http://www.w3.org/2000/01/rdf-schema\#comment}
  \end{itemize}
\item
  \textbf{\texttt{canonical\_order}} (\texttt{integer}): Sorting order
  for visual presentation. The order in which items are canonically
  presented. Use integers to represent relative order, order is used
  per-item.

  \begin{itemize}
  \tightlist
  \item
    \texttt{rdfProperty}:
    \url{https://www.paralex-standard.org/paralex_ontology.xml\#canonical_order}
  \end{itemize}
\end{itemize}

\hypertarget{cells}{%
\subsection{\texorpdfstring{\textbf{cells}}{cells}}\label{cells}}

Paradigm cells - This file is located in \texttt{cells.csv}.

\begin{itemize}
\tightlist
\item
  The identifier column (or \texttt{primaryKey}) is
  \texttt{{[}\textquotesingle{}cell\_id\textquotesingle{}{]}}
\end{itemize}

\textbf{Columns} defined by \texttt{cells-schema}:

\begin{itemize}
\item
  \textbf{\texttt{cell\_id}} (\texttt{string}): Cell identifier. The set
  of feature values as would appear in a gloss, separated by dots, eg.
  prs.ind.1sg or f.pl

  \begin{itemize}
  \tightlist
  \item
    constraints: a \texttt{cell\_id} is obligatory; it must be unique.
  \end{itemize}
\item
  \textbf{\texttt{label}} (\texttt{string}): label for this row. A human
  readable label for the row.

  \begin{itemize}
  \tightlist
  \item
    \texttt{rdfProperty}:
    \url{http://www.w3.org/2000/01/rdf-schema\#label}
  \end{itemize}
\item
  \textbf{\texttt{unimorph}} (\texttt{string}): Cell in unimorph format.
  The cell, written following the unimorph schema

  \begin{itemize}
  \tightlist
  \item
    \texttt{rdfProperty}:
    \url{https://www.paralex-standard.org/paralex_ontology.xml\#unimorph}
  \end{itemize}
\item
  \textbf{\texttt{ud}} (\texttt{string}): Cell in the universal
  dependency format. The cell, written following the universal
  dependency format

  \begin{itemize}
  \tightlist
  \item
    \texttt{rdfProperty}:
    \url{https://www.paralex-standard.org/paralex_ontology.xml\#ud}
  \end{itemize}
\item
  \textbf{\texttt{comment}} (\texttt{string}): Comment. Human-readable
  comment.

  \begin{itemize}
  \tightlist
  \item
    \texttt{rdfProperty}:
    \url{http://www.w3.org/2000/01/rdf-schema\#comment}
  \end{itemize}
\item
  \textbf{\texttt{POS}} (\texttt{string}): Part of Speech. The relevant
  part of speech for this item. This must refer to a PartOfSpeech entity
  from the lexinfo (https://lexinfo.net/) ontology.

  \begin{itemize}
  \item
    constraints: a \texttt{POS} must be one of the values:
    \texttt{verb}, \texttt{numeral}, \texttt{conjunction},
    \texttt{noun}, \texttt{adposition}, \texttt{determiner},
    \texttt{article}, \texttt{adverb}, \texttt{pronoun},
    \texttt{fusedPreposition}, \texttt{adjective}, \texttt{symbol},
    \texttt{particle}, \texttt{conditionalParticle},
    \texttt{demonstrativePronoun}, \texttt{interjection},
    \texttt{semiColon}, \texttt{diminutiveNoun},
    \texttt{possessivePronoun}, \texttt{prepositionalAdverb},
    \texttt{compoundPreposition}, \texttt{interrogativeRelativePronoun},
    \texttt{possessiveParticle}, \texttt{plainVerb}, \texttt{letter},
    \texttt{interrogativeDeterminer}, \texttt{relativePronoun},
    \texttt{postposition}, \texttt{fusedPronounAuxiliary},
    \texttt{interrogativeOrdinalNumeral},
    \texttt{indefiniteOrdinalNumeral}, \texttt{strongPersonalPronoun},
    \texttt{possessiveRelativePronoun}, \texttt{ordinalAdjective},
    \texttt{collectivePronoun}, \texttt{commonNoun},
    \texttt{infinitiveParticle}, \texttt{comparativeParticle},
    \texttt{partitiveArticle}, \texttt{invertedComma},
    \texttt{lightVerb}, \texttt{emphaticPronoun},
    \texttt{distinctiveParticle}, \texttt{genericNumeral},
    \texttt{possessiveAdjective}, \texttt{reflexivePossessivePronoun},
    \texttt{colon}, \texttt{coordinationParticle},
    \texttt{presentParticipleAdjective},
    \texttt{fusedPrepositionPronoun}, \texttt{cardinalNumeral},
    \texttt{indefiniteDeterminer}, \texttt{numeralFraction},
    \texttt{questionMark}, \texttt{generalAdverb},
    \texttt{superlativeParticle}, \texttt{point},
    \texttt{indefiniteMultiplicativeNumeral}, \texttt{comma},
    \texttt{closeParenthesis}, \texttt{futureParticle},
    \texttt{personalPronoun}, \texttt{reflexivePersonalPronoun},
    \texttt{adverbialPronoun}, \texttt{reciprocalPronoun},
    \texttt{openParenthesis}, \texttt{pastParticipleAdjective},
    \texttt{negativePronoun}, \texttt{relativeDeterminer},
    \texttt{existentialPronoun}, \texttt{pronominalAdverb},
    \texttt{relativeParticle}, \texttt{exclamativeDeterminer},
    \texttt{multiplicativeNumeral}, \texttt{reflexiveDeterminer},
    \texttt{modal}, \texttt{unclassifiedParticle}, \texttt{properNoun},
    \texttt{allusivePronoun}, \texttt{interrogativeCardinalNumeral},
    \texttt{bullet}, \texttt{subordinatingConjunction},
    \texttt{irreflexivePersonalPronoun}, \texttt{possessiveDeterminer},
    \texttt{negativeParticle}, \texttt{indefinitePronoun},
    \texttt{generalizationWord}, \texttt{coordinatingConjunction},
    \texttt{deficientVerb}, \texttt{adjective-i},
    \texttt{impersonalPronoun}, \texttt{indefiniteCardinalNumeral},
    \texttt{adjective-na}, \texttt{qualifierAdjective},
    \texttt{affirmativeParticle}, \texttt{mainVerb},
    \texttt{fusedPrepositionDeterminer}, \texttt{indefiniteArticle},
    \texttt{weakPersonalPronoun}, \texttt{suspensionPoints},
    \texttt{interrogativeMultiplicativeNumeral},
    \texttt{affixedPersonalPronoun}, \texttt{auxiliary},
    \texttt{circumposition}, \texttt{copula},
    \texttt{demonstrativeDeterminer}, \texttt{participleAdjective},
    \texttt{exclamativePoint}, \texttt{interrogativePronoun},
    \texttt{presentativePronoun}, \texttt{punctuation},
    \texttt{definiteArticle}, \texttt{slash},
    \texttt{exclamativePronoun}, \texttt{preposition},
    \texttt{conditionalPronoun}, \texttt{relationNoun},
    \texttt{interrogativeParticle}.
  \item
    \texttt{rdfProperty}:
    \url{https://www.paralex-standard.org/paralex_ontology.xml\#POS}
  \end{itemize}
\item
  \textbf{\texttt{frequency}} (\texttt{number}): Frequency. Frequency
  for this row.

  \begin{itemize}
  \tightlist
  \item
    \texttt{rdfProperty}:
    \url{https://www.paralex-standard.org/paralex_ontology.xml\#frequency}
  \end{itemize}
\item
  \textbf{\texttt{canonical\_order}} (\texttt{integer}): Sorting order
  for visual presentation. The order in which items are canonically
  presented. Use integers to represent relative order, order is used
  per-item.

  \begin{itemize}
  \tightlist
  \item
    \texttt{rdfProperty}:
    \url{https://www.paralex-standard.org/paralex_ontology.xml\#canonical_order}
  \end{itemize}
\end{itemize}

\hypertarget{features-values}{%
\subsection{\texorpdfstring{\textbf{features-values}}{features-values}}\label{features-values}}

Grammatical features values - This file is located in
\texttt{features-values.csv}.

\begin{itemize}
\tightlist
\item
  The identifier column (or \texttt{primaryKey}) is
  \texttt{{[}\textquotesingle{}value\_id\textquotesingle{}{]}}
\end{itemize}

\textbf{Columns} defined by \texttt{features-values-schema}:

\begin{itemize}
\item
  \textbf{\texttt{value\_id}} (\texttt{string}): Grammatical Feature
  value identifier. Identifier for the grammatical feature value (as
  found in the cell)

  \begin{itemize}
  \tightlist
  \item
    constraints: a \texttt{value\_id} is obligatory; it must be unique.
  \end{itemize}
\item
  \textbf{\texttt{label}} (\texttt{string}): label for this row. A human
  readable label for the row.

  \begin{itemize}
  \tightlist
  \item
    \texttt{rdfProperty}:
    \url{http://www.w3.org/2000/01/rdf-schema\#label}
  \end{itemize}
\item
  \textbf{\texttt{feature}} (\texttt{string}): feature. The name of the
  dimension of this feature, eg. case, tense, modality, voice, force,
  gender, evidentiality, person, number, polarity\ldots{}

  \begin{itemize}
  \item
    constraints: a \texttt{feature} is obligatory.
  \item
    \texttt{rdfProperty}:
    \url{https://www.paralex-standard.org/paralex_ontology.xml\#feature}
  \end{itemize}
\item
  \textbf{\texttt{comment}} (\texttt{string}): Comment. Human-readable
  comment.

  \begin{itemize}
  \tightlist
  \item
    \texttt{rdfProperty}:
    \url{http://www.w3.org/2000/01/rdf-schema\#comment}
  \end{itemize}
\item
  \textbf{\texttt{POS}} (\texttt{string}): Part of Speech. The relevant
  part of speech for this item. This must refer to a PartOfSpeech entity
  from the lexinfo (https://lexinfo.net/) ontology.

  \begin{itemize}
  \item
    constraints: a \texttt{POS} must be one of the values:
    \texttt{verb}, \texttt{numeral}, \texttt{conjunction},
    \texttt{noun}, \texttt{adposition}, \texttt{determiner},
    \texttt{article}, \texttt{adverb}, \texttt{pronoun},
    \texttt{fusedPreposition}, \texttt{adjective}, \texttt{symbol},
    \texttt{particle}, \texttt{conditionalParticle},
    \texttt{demonstrativePronoun}, \texttt{interjection},
    \texttt{semiColon}, \texttt{diminutiveNoun},
    \texttt{possessivePronoun}, \texttt{prepositionalAdverb},
    \texttt{compoundPreposition}, \texttt{interrogativeRelativePronoun},
    \texttt{possessiveParticle}, \texttt{plainVerb}, \texttt{letter},
    \texttt{interrogativeDeterminer}, \texttt{relativePronoun},
    \texttt{postposition}, \texttt{fusedPronounAuxiliary},
    \texttt{interrogativeOrdinalNumeral},
    \texttt{indefiniteOrdinalNumeral}, \texttt{strongPersonalPronoun},
    \texttt{possessiveRelativePronoun}, \texttt{ordinalAdjective},
    \texttt{collectivePronoun}, \texttt{commonNoun},
    \texttt{infinitiveParticle}, \texttt{comparativeParticle},
    \texttt{partitiveArticle}, \texttt{invertedComma},
    \texttt{lightVerb}, \texttt{emphaticPronoun},
    \texttt{distinctiveParticle}, \texttt{genericNumeral},
    \texttt{possessiveAdjective}, \texttt{reflexivePossessivePronoun},
    \texttt{colon}, \texttt{coordinationParticle},
    \texttt{presentParticipleAdjective},
    \texttt{fusedPrepositionPronoun}, \texttt{cardinalNumeral},
    \texttt{indefiniteDeterminer}, \texttt{numeralFraction},
    \texttt{questionMark}, \texttt{generalAdverb},
    \texttt{superlativeParticle}, \texttt{point},
    \texttt{indefiniteMultiplicativeNumeral}, \texttt{comma},
    \texttt{closeParenthesis}, \texttt{futureParticle},
    \texttt{personalPronoun}, \texttt{reflexivePersonalPronoun},
    \texttt{adverbialPronoun}, \texttt{reciprocalPronoun},
    \texttt{openParenthesis}, \texttt{pastParticipleAdjective},
    \texttt{negativePronoun}, \texttt{relativeDeterminer},
    \texttt{existentialPronoun}, \texttt{pronominalAdverb},
    \texttt{relativeParticle}, \texttt{exclamativeDeterminer},
    \texttt{multiplicativeNumeral}, \texttt{reflexiveDeterminer},
    \texttt{modal}, \texttt{unclassifiedParticle}, \texttt{properNoun},
    \texttt{allusivePronoun}, \texttt{interrogativeCardinalNumeral},
    \texttt{bullet}, \texttt{subordinatingConjunction},
    \texttt{irreflexivePersonalPronoun}, \texttt{possessiveDeterminer},
    \texttt{negativeParticle}, \texttt{indefinitePronoun},
    \texttt{generalizationWord}, \texttt{coordinatingConjunction},
    \texttt{deficientVerb}, \texttt{adjective-i},
    \texttt{impersonalPronoun}, \texttt{indefiniteCardinalNumeral},
    \texttt{adjective-na}, \texttt{qualifierAdjective},
    \texttt{affirmativeParticle}, \texttt{mainVerb},
    \texttt{fusedPrepositionDeterminer}, \texttt{indefiniteArticle},
    \texttt{weakPersonalPronoun}, \texttt{suspensionPoints},
    \texttt{interrogativeMultiplicativeNumeral},
    \texttt{affixedPersonalPronoun}, \texttt{auxiliary},
    \texttt{circumposition}, \texttt{copula},
    \texttt{demonstrativeDeterminer}, \texttt{participleAdjective},
    \texttt{exclamativePoint}, \texttt{interrogativePronoun},
    \texttt{presentativePronoun}, \texttt{punctuation},
    \texttt{definiteArticle}, \texttt{slash},
    \texttt{exclamativePronoun}, \texttt{preposition},
    \texttt{conditionalPronoun}, \texttt{relationNoun},
    \texttt{interrogativeParticle}.
  \item
    \texttt{rdfProperty}:
    \url{https://www.paralex-standard.org/paralex_ontology.xml\#POS}
  \end{itemize}
\item
  \textbf{\texttt{unimorph}} (\texttt{string}): Cell in unimorph format.
  The cell, written following the unimorph schema

  \begin{itemize}
  \tightlist
  \item
    \texttt{rdfProperty}:
    \url{https://www.paralex-standard.org/paralex_ontology.xml\#unimorph}
  \end{itemize}
\item
  \textbf{\texttt{ud}} (\texttt{string}): Cell in the universal
  dependency format. The cell, written following the universal
  dependency format

  \begin{itemize}
  \tightlist
  \item
    \texttt{rdfProperty}:
    \url{https://www.paralex-standard.org/paralex_ontology.xml\#ud}
  \end{itemize}
\item
  \textbf{\texttt{canonical\_order}} (\texttt{integer}): Sorting order
  for visual presentation. The order in which items are canonically
  presented. Use integers to represent relative order, order is used
  per-item.

  \begin{itemize}
  \tightlist
  \item
    \texttt{rdfProperty}:
    \url{https://www.paralex-standard.org/paralex_ontology.xml\#canonical_order}
  \end{itemize}
\end{itemize}

\hypertarget{lexemes}{%
\subsection{\texorpdfstring{\textbf{lexemes}}{lexemes}}\label{lexemes}}

Lexemes - This file is located in \texttt{lexemes.csv}.

\begin{itemize}
\tightlist
\item
  The identifier column (or \texttt{primaryKey}) is
  \texttt{{[}\textquotesingle{}lexeme\_id\textquotesingle{}{]}}
\end{itemize}

\textbf{Columns} defined by \texttt{lexemes-schema}:

\begin{itemize}
\item
  \textbf{\texttt{lexeme\_id}} (\texttt{string}): Identifier for the
  lexeme. Lexeme identifiers. Often, they are identical to the label
  (lemma). However, they must be unique to paradigms, distinguishing
  homonyms with different inflection. For example, the animal mouse/mice
  and the computer peripheric mouse/mouses would both have the label
  `mouse' but could be identified by the lexeme identifiers mouse\_1 and
  mouse\_2.

  \begin{itemize}
  \tightlist
  \item
    constraints: a \texttt{lexeme\_id} is obligatory; it must be unique.
  \end{itemize}
\item
  \textbf{\texttt{inflection\_class}} (\texttt{string}): Inflection
  class identifier. This identifier groups together lexemes of the same
  inflection class.

  \begin{itemize}
  \tightlist
  \item
    \texttt{rdfProperty}:
    \url{https://www.paralex-standard.org/paralex_ontology.xml\#inflection_class}
  \end{itemize}
\item
  \textbf{\texttt{source}} (\texttt{string}): Source. Reference to a
  specific source (bibtex key). If used, the dataset should comprise a
  .bib file where the keys are referenced.

  \begin{itemize}
  \tightlist
  \item
    \texttt{rdfProperty}:
    \url{https://www.paralex-standard.org/paralex_ontology.xml\#source}
  \end{itemize}
\item
  \textbf{\texttt{frequency}} (\texttt{number}): Frequency. Frequency
  for this row.

  \begin{itemize}
  \tightlist
  \item
    \texttt{rdfProperty}:
    \url{https://www.paralex-standard.org/paralex_ontology.xml\#frequency}
  \end{itemize}
\item
  \textbf{\texttt{label}} (\texttt{string}): label for this row. A human
  readable label for the row.

  \begin{itemize}
  \tightlist
  \item
    \texttt{rdfProperty}:
    \url{http://www.w3.org/2000/01/rdf-schema\#label}
  \end{itemize}
\item
  \textbf{\texttt{meaning}} (\texttt{string}): Definition for this
  lexeme. This is a description of the lexeme's overall meaning.

  \begin{itemize}
  \tightlist
  \item
    \texttt{rdfProperty}:
    \url{https://www.paralex-standard.org/paralex_ontology.xml\#meaning}
  \end{itemize}
\item
  \textbf{\texttt{gloss}} (\texttt{string}): Short meaning, used for
  glossing.. Gloss for this lexeme.

  \begin{itemize}
  \tightlist
  \item
    \texttt{rdfProperty}:
    \url{https://www.paralex-standard.org/paralex_ontology.xml\#gloss}
  \end{itemize}
\item
  \textbf{\texttt{comment}} (\texttt{string}): Comment. Human-readable
  comment.

  \begin{itemize}
  \tightlist
  \item
    \texttt{rdfProperty}:
    \url{http://www.w3.org/2000/01/rdf-schema\#comment}
  \end{itemize}
\item
  \textbf{\texttt{POS}} (\texttt{string}): Part of Speech. The relevant
  part of speech for this item. This must refer to a PartOfSpeech entity
  from the lexinfo (https://lexinfo.net/) ontology.

  \begin{itemize}
  \item
    constraints: a \texttt{POS} must be one of the values:
    \texttt{verb}, \texttt{numeral}, \texttt{conjunction},
    \texttt{noun}, \texttt{adposition}, \texttt{determiner},
    \texttt{article}, \texttt{adverb}, \texttt{pronoun},
    \texttt{fusedPreposition}, \texttt{adjective}, \texttt{symbol},
    \texttt{particle}, \texttt{conditionalParticle},
    \texttt{demonstrativePronoun}, \texttt{interjection},
    \texttt{semiColon}, \texttt{diminutiveNoun},
    \texttt{possessivePronoun}, \texttt{prepositionalAdverb},
    \texttt{compoundPreposition}, \texttt{interrogativeRelativePronoun},
    \texttt{possessiveParticle}, \texttt{plainVerb}, \texttt{letter},
    \texttt{interrogativeDeterminer}, \texttt{relativePronoun},
    \texttt{postposition}, \texttt{fusedPronounAuxiliary},
    \texttt{interrogativeOrdinalNumeral},
    \texttt{indefiniteOrdinalNumeral}, \texttt{strongPersonalPronoun},
    \texttt{possessiveRelativePronoun}, \texttt{ordinalAdjective},
    \texttt{collectivePronoun}, \texttt{commonNoun},
    \texttt{infinitiveParticle}, \texttt{comparativeParticle},
    \texttt{partitiveArticle}, \texttt{invertedComma},
    \texttt{lightVerb}, \texttt{emphaticPronoun},
    \texttt{distinctiveParticle}, \texttt{genericNumeral},
    \texttt{possessiveAdjective}, \texttt{reflexivePossessivePronoun},
    \texttt{colon}, \texttt{coordinationParticle},
    \texttt{presentParticipleAdjective},
    \texttt{fusedPrepositionPronoun}, \texttt{cardinalNumeral},
    \texttt{indefiniteDeterminer}, \texttt{numeralFraction},
    \texttt{questionMark}, \texttt{generalAdverb},
    \texttt{superlativeParticle}, \texttt{point},
    \texttt{indefiniteMultiplicativeNumeral}, \texttt{comma},
    \texttt{closeParenthesis}, \texttt{futureParticle},
    \texttt{personalPronoun}, \texttt{reflexivePersonalPronoun},
    \texttt{adverbialPronoun}, \texttt{reciprocalPronoun},
    \texttt{openParenthesis}, \texttt{pastParticipleAdjective},
    \texttt{negativePronoun}, \texttt{relativeDeterminer},
    \texttt{existentialPronoun}, \texttt{pronominalAdverb},
    \texttt{relativeParticle}, \texttt{exclamativeDeterminer},
    \texttt{multiplicativeNumeral}, \texttt{reflexiveDeterminer},
    \texttt{modal}, \texttt{unclassifiedParticle}, \texttt{properNoun},
    \texttt{allusivePronoun}, \texttt{interrogativeCardinalNumeral},
    \texttt{bullet}, \texttt{subordinatingConjunction},
    \texttt{irreflexivePersonalPronoun}, \texttt{possessiveDeterminer},
    \texttt{negativeParticle}, \texttt{indefinitePronoun},
    \texttt{generalizationWord}, \texttt{coordinatingConjunction},
    \texttt{deficientVerb}, \texttt{adjective-i},
    \texttt{impersonalPronoun}, \texttt{indefiniteCardinalNumeral},
    \texttt{adjective-na}, \texttt{qualifierAdjective},
    \texttt{affirmativeParticle}, \texttt{mainVerb},
    \texttt{fusedPrepositionDeterminer}, \texttt{indefiniteArticle},
    \texttt{weakPersonalPronoun}, \texttt{suspensionPoints},
    \texttt{interrogativeMultiplicativeNumeral},
    \texttt{affixedPersonalPronoun}, \texttt{auxiliary},
    \texttt{circumposition}, \texttt{copula},
    \texttt{demonstrativeDeterminer}, \texttt{participleAdjective},
    \texttt{exclamativePoint}, \texttt{interrogativePronoun},
    \texttt{presentativePronoun}, \texttt{punctuation},
    \texttt{definiteArticle}, \texttt{slash},
    \texttt{exclamativePronoun}, \texttt{preposition},
    \texttt{conditionalPronoun}, \texttt{relationNoun},
    \texttt{interrogativeParticle}.
  \item
    \texttt{rdfProperty}:
    \url{https://www.paralex-standard.org/paralex_ontology.xml\#POS}
  \end{itemize}
\item
  \textbf{\texttt{language\_ID}} (\texttt{string}): Identifier for the
  language. Language identifiers should use some standard ID (iso code,
  glottocode, etc)

  \begin{itemize}
  \tightlist
  \item
    \texttt{rdfProperty}:
    \url{https://www.paralex-standard.org/paralex_ontology.xml\#language_ID}
  \end{itemize}
\item
  \textbf{\texttt{analysis\_tag}} (\texttt{string}): Tags for marking
  separate analyses. Identifies sets of forms which are related by the
  same analysis. Eg: forms from two distinct sources, or a more phonetic
  and a more phonological transcription.

  \begin{itemize}
  \tightlist
  \item
    \texttt{rdfProperty}:
    \url{https://www.paralex-standard.org/paralex_ontology.xml\#analysis_tag}
  \end{itemize}
\item
  \textbf{\texttt{defectiveness\_tag}} (\texttt{string}): Tags for
  defectiveness status. Identifies sets of defective forms (eg. pluralia
  tantum).

  \begin{itemize}
  \tightlist
  \item
    \texttt{rdfProperty}:
    \url{https://www.paralex-standard.org/paralex_ontology.xml\#defectiveness_tag}
  \end{itemize}
\item
  \textbf{\texttt{epistemic\_tag}} (\texttt{string}): Tags for epistemic
  status. Identifies sets of forms with the same epistemic status.

  \begin{itemize}
  \tightlist
  \item
    \texttt{rdfProperty}:
    \url{https://www.paralex-standard.org/paralex_ontology.xml\#epistemic_tag}
  \end{itemize}
\item
  \textbf{\texttt{variants\_tag}} (\texttt{string}): Tags for form
  variants. Identifies sets of forms used by specific groups of
  speakers. Eg. dialectal variants.

  \begin{itemize}
  \tightlist
  \item
    \texttt{rdfProperty}:
    \url{https://www.paralex-standard.org/paralex_ontology.xml\#variants_tag}
  \end{itemize}
\item
  \textbf{\texttt{overabundance\_tag}} (\texttt{string}): Tags for
  overabundant forms. Identifies sets of overabundant forms. For
  example, overabundant forms across lexemes might belong to a series of
  regular and irregular forms, or a series of short and long forms, etc.

  \begin{itemize}
  \tightlist
  \item
    \texttt{rdfProperty}:
    \url{https://www.paralex-standard.org/paralex_ontology.xml\#overabundance_tag}
  \end{itemize}
\end{itemize}

\hypertarget{tags}{%
\subsection{\texorpdfstring{\textbf{tags}}{tags}}\label{tags}}

Tags mark rows which have commonalities - This file is located in
\texttt{tags.csv}.

\begin{itemize}
\tightlist
\item
  The identifier column (or \texttt{primaryKey}) is
  \texttt{{[}\textquotesingle{}tag\_id\textquotesingle{}{]}}
\end{itemize}

\textbf{Columns} defined by \texttt{tags-schema}:

\begin{itemize}
\item
  \textbf{\texttt{tag\_id}} (\texttt{string}): Tag id. The label for a
  set of forms which have something in common.

  \begin{itemize}
  \tightlist
  \item
    constraints: a \texttt{tag\_id} is obligatory; it must be unique.
  \end{itemize}
\item
  \textbf{\texttt{tag\_column\_name}} (\texttt{string}): Name of the tag
  column in the forms table. The name of the column this tag is used in
  the forms table

  \begin{itemize}
  \tightlist
  \item
    constraints: a \texttt{tag\_column\_name} is obligatory; it must
    match the regular expression \texttt{{[}\^{}\ {]}+\_tag}.
  \end{itemize}
\item
  \textbf{\texttt{comment}} (\texttt{string}): Comment. Human-readable
  comment.

  \begin{itemize}
  \tightlist
  \item
    \texttt{rdfProperty}:
    \url{http://www.w3.org/2000/01/rdf-schema\#comment}
  \end{itemize}
\end{itemize}

\hypertarget{sources}{%
\subsection{\texorpdfstring{\textbf{sources}}{sources}}\label{sources}}

Sources\textbar{} Bibliographical references.

\begin{itemize}
\tightlist
\item
  This file is located in \texttt{sources.bib}.
\end{itemize}

\hypertarget{readme}{%
\subsection{\texorpdfstring{\textbf{readme}}{readme}}\label{readme}}

Read me\textbar{} Basic documentation

\begin{itemize}
\tightlist
\item
  This file is located in \texttt{readme.md}.
\end{itemize}

\hypertarget{data_sheet}{%
\subsection{\texorpdfstring{\textbf{data\_sheet}}{data\_sheet}}\label{data_sheet}}

Data Sheet\textbar{} Data Sheet

\begin{itemize}
\tightlist
\item
  This file is located in \texttt{data\_sheet.md}.
\end{itemize}
